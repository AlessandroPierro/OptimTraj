\documentclass[onecolumn]{article}
%%%% Compile with PdfLaTex

\usepackage{graphicx} % handles graphics and figures
\usepackage{mathtools,bm}
\usepackage{bbold}   %Get fancy double struck math notation for sets
\usepackage{courier}   %Have code written out nicely
\usepackage{color}
\usepackage{algorithm}
\usepackage[noend]{algpseudocode}
\usepackage{float}
\usepackage[top=1in, bottom=1in, left=1in, right=1in]{geometry}
\usepackage[titletoc,title]{appendix}
\usepackage{hyperref}

\catcode`\^^M=10      %  Makes blank lines meaningless, force use of \par 

%Custon inline font commands
\newcommand{\quotes}[1]{``#1''}   % Quotes!
\newcommand{\eqnTxt}[1]{\text{\footnotesize \textbf{#1}} }   % annotation in an equation

% Custom editing commands                 
\definecolor{lightblue}{rgb}{0.0,0.5,0.8}
\newcommand{\todo}[1]{{\color{lightblue}\par {[{\bf TO DO: } {\em #1}} ] \\    }}

% Special formatting for struct field names:
\definecolor{darkgreen}{rgb}{0.0, 0.4, 0.2}
\newcommand{\tc}[1]{{\bf \texttt{#1}}}  % Soft-coded variable names (can be changed)
\newcommand{\hc}[1]{{\color{darkgreen}{\bf \texttt{#1}}}}  % Hard-coded variable names

%========================================================================
\title{TrajOpt Users Guide  \\  Version 1.5}
\author{Matthew P. Kelly}

%====================================================
%====================================================
\begin{document} % 
\maketitle

\section{Introduction}

OptimTraj is a matlab library designed for solving continuous-time single-phase trajectory optimization problems. I developed it while working on my PhD at Cornell, studying non-linear controller design for walking robots.

\subsection{What sort of problems does OptimTraj solve?}

\subsubsection*{Examples:}
\begin{itemize} 
\item Cart-Pole Swing-Up: Find the force profile to apply to the cart to swing-up the pendulum that freely hangs from it.
\item Compute the gait (joint angles, rates, and torques) for a walking robot that minimizes the energy used while walking.
\item Find a minimum-thrust orbit transfer trajectory for a satellite.
\end{itemize}

\subsubsection*{Details:}

OptimTraj finds the optimal trajectory for a dynamical system. This trajectory is a sequence of controls (expressed as a function) that moves the dynamical system between two points in state space. The trajectory will minimize some cost function, which is typically an integral along the trajectory. The trajectory will also satisfy a set user-defined constraints. All functions in the problem description can be non-linear, but they must be smooth (C$^2$ continuous).

OptimTraj solves problems with
\begin{itemize} \setlength\itemsep{-0.1em}
\item continuous dynamics
\item boundary constraints
\item path constraints
\item integral cost function
\item boundary cost function
\end{itemize}


\subsection{Features:}
\begin{itemize} \setlength\itemsep{-0.1em}
\item {\bf Easy to Install: } no dependencies outside of Matlab \footnotemark
\item {\bf Lots of example: } look at the \texttt{deom/} directory to see for yourself!
\item {\bf Readable source code: } easy to debug your code and figure out how the software works
\item {\bf Analytic gradients: } most methods support analytic gradients
\item {\bf Rapidly switch between methods: } helpful for debugging and experimenting
 	\begin{itemize} \setlength\itemsep{-0.1em}
 	\item Trapezoidal Direct Collocation
 	\item Hermite-Simpson Direct Collocation
 	\item Runge--Kutta $4^\text{th}$-order Multiple Shooting
 	\item Chebyshev--Lobatto Orthogonal Collocation 
 	\end{itemize}
\end{itemize}
\footnotetext{Chebyshev--Lobatto method requires ChebFun (\texttt{http://www.chebfun.org})}

\subsection{Installation:}
\begin{enumerate} \setlength\itemsep{-0.1em}
\item Clone or download the repository (\texttt{https://github.com/MatthewPeterKelly/OptimTraj})
\item Add the top level folder to your Matlab path
\item (Optional) Clone or download ChebFun (\texttt{http://www.chebfun.org}) to use the Chebyshev--Lobatto method
\end{enumerate}

\subsection{Usage: }
\begin{itemize} \setlength\itemsep{-0.1em}
\item Call the function \texttt{optimTraj} from inside matlab.
\item \texttt{optimTraj} takes a single argument: a struct that describes your trajectory optimization problem.
\item \texttt{optimTraj} returns a struct that describes the solution. It contains a full description of the problem, the transcription method that was used, and the solution (both as a vector of points and a function handle for interpolation).
\item For more details, type \texttt{>> help optimTraj} at the command line, or check out some of the examples in the \texttt{demo/} directory.
\end{itemize}

\subsection{License: }
OptimTraj is published under the MIT License, Copyright (c) 2016 Matthew P. Kelly.  \vspace{0.5em} \\

Permission is hereby granted, free of charge, to any person obtaining a copy
of this software and associated documentation files (the "Software"), to deal
in the Software without restriction, including without limitation the rights
to use, copy, modify, merge, publish, distribute, sublicense, and/or sell
copies of the Software, and to permit persons to whom the Software is
furnished to do so, subject to the following conditions:  \vspace{0.5em} \\

The above copyright notice and this permission notice shall be included in
all copies or substantial portions of the Software.  \vspace{0.5em} \\

THE SOFTWARE IS PROVIDED "AS IS", WITHOUT WARRANTY OF ANY KIND, EXPRESS OR
IMPLIED, INCLUDING BUT NOT LIMITED TO THE WARRANTIES OF MERCHANTABILITY,
FITNESS FOR A PARTICULAR PURPOSE AND NONINFRINGEMENT. IN NO EVENT SHALL THE
AUTHORS OR COPYRIGHT HOLDERS BE LIABLE FOR ANY CLAIM, DAMAGES OR OTHER
LIABILITY, WHETHER IN AN ACTION OF CONTRACT, TORT OR OTHERWISE, ARISING FROM,
OUT OF OR IN CONNECTION WITH THE SOFTWARE OR THE USE OR OTHER DEALINGS IN
THE SOFTWARE.


\section{Using TrajOpt}

There is a single calling sequence for using TrajOpt:  \tc{soln = }\hc{trajOpt}\tc{(problem)}. The input \tc{problem} is a struct that describes a trajectory optimization problem, and the output \tc{soln} is a struct that gives details regarding the solution to the trajectory optimization problem. This section is similar to the help file for \tc{trajOpt()}.

\subsection{Notation}

Throughout this section we will use $t$ for time, $\bm{x}$ for state, and $\bm{u}$ for control. We will use $N$ for an integer, where $N_t$ is the number of grid-points along the trajectory and $N_g$ is the number of grid-points in the initial guess. The dimension of the state is given by $N_x$ and the dimension of the control is given by $N_u$. We will use this notation: \tc{problem.}\hc{func} to indicate that the field \hc{func} must be typed exactly as shown, while the struct \tc{problem} can be named anything.

\par We will use $t_0$ and $t_F$ to indicate the initial and final times. Similarly, $\bm{x}_0 = \bm{x}(t_0)$ and $\bm{x}_F = \bm{x}(t_F)$ are the initial and final state. Throughout the software, we use the convention that time is a row vector, while the dimensions of state and control are a column vector. Thus, \\
\texttt{size($t$) = [1, $N_t$]}, \\
\texttt{size($\bm{x}$) = [$N_x$, $N_t$]}, \\
\texttt{size($\bm{u}$) = [$N_u$, $N_t$]}, \\
\texttt{size($\bm{x}_0$) = size($\bm{x}_F$)  = [$N_x$, 1]}. 


\subsection{Constructing the Input to TrajOpt}

The input to \hc{trajOpt()} is a single struct, which we will call the \tc{problem} struct. The \tc{problem} struct has four fields. The \tc{problem}.\hc{func} struct contains a set of user-defined function handles to the dynamics, objective, and constraint functions. The \tc{problem}.\hc{bounds} struct contains all constant bounds on trajectory (time, state, and control). The \tc{problem}.\hc{guess} struct contains an initial guess for the trajectory. Finally, the \tc{problem}.\hc{options} struct contains options for both \hc{trajOpt()} and \hc{fmincon()}, which it calls to solve the underlying non-linear program.

\subsubsection*{\tc{problem}.\hc{func}}

There are five fields in the \tc{problem}.\hc{func} struct, each of which is a user-defined function handle. The only mandatory field is \texttt{func.dynamics}, and at least one of either \texttt{func.pathObj} or \texttt{func.bndObj}. All other fields can simply be omitted or left empty \texttt{[]}. Here we will list the prototype for each function handle: \texttt{outArgs = funName(inArgs)}. The user can pass additional parameters using the Matlab anonymous function syntax: \texttt{funHandle = @( funName(inArgs, extraParams) );}.

\par Both of the direct collocation methods (trapezoidal and Hermite-Simpson) support analytic gradients. The details regarding the construction of these function handles is provided in the help files for each of these methods: \texttt{>> help trapezoid} and \texttt{>> help hermiteSimpson}. If you're using analytic gradients, then you should look at the \texttt{demo/gradientsTutorial} for an simple tutorial example and \texttt{demo/fiveLinkBiped} for a complicated example.

\begin{itemize} \setlength\itemsep{-0.1em}
\item \tc{problem.}\hc{func.dynamics}
	${\bm: \quad \to \quad }$
	\tc{$\bm{\dot{x}}$ = }\hc{dynamics}\tc{($t$,$\bm{x}$,$\bm{u}$)}
	\\ where $\bm{\dot{x}}$ is the time derivative of $\bm{x}$ and \texttt{size($\bm{\dot{x}}$)==size($\bm{x}$)}
\item \tc{problem.}\hc{func.pathObj}
	${\bm: \quad \to \quad }$
	\tc{$\bm{J_P}$ = }\hc{pathObj}\tc{($t$,$\bm{x}$,$\bm{u}$)}
	\\ where $\bm{J_P}$ is the integrand of the objective function and \texttt{size($\bm{J_P}$)==size($t$)}
\item \tc{problem.}\hc{func.pathCst}
	${\bm: \quad \to \quad }$
	\tc{$\bm{[C_P^a, C_P^b ]}$ = }\hc{pathCst}\tc{($t$,$\bm{x}$,$\bm{u}$)}
	\\ where $\bm{C_P^a}=\bm{0}$ is path equality constraint, and $\bm{C_P^b} \leq \bm{0}$ is the path inequality constraint. Either may be left empty \texttt{[ ]}. 	 	The 	number of columns in each must equal that of time, but the number of rows in each is arbitrary (it just must be consistent between function calls).
\item \tc{problem.}\hc{func.bndObj}
	${\bm: \quad \to \quad }$
	\tc{$J_B$ = }\hc{bndObj}\tc{($t_0$,$\bm{x}_0$,$t_F$,$\bm{x}_F$)}
	\\ where $\bm{J_B}$ is a scalar cost associated with the boundary points of the trajectory. 
\item \tc{problem.}\hc{func.bndCst}
	${\bm: \quad \to \quad }$
	\tc{$\bm{[C_B^a, C_B^b] }$ = }\hc{bndCst}\tc{($t_0$,$\bm{x}_0$,$t_F$,$\bm{x}_F$)}
	\\ where $\bm{C_B^a}=\bm{0}$ is boundary equality constraint, and $\bm{C_B^b} \leq \bm{0}$ is the boundary inequality constraint. Either may be left empty \texttt{[ ]}. Each is a column vector of arbitrary length, provided that it is consistent between function calls.
\end{itemize}


\subsubsection*{\tc{problem}.\hc{bounds}}

The bounds struct provides constant bounds on the state and control along the trajectory, as well as the time and state on the boundaries. All fields are either scalar or a column vector, and can be omitted (or left empty) if not needed. If you need to include a bound on only part of the state or control, then set the remaining entries to $\pm\infty$. For example: \hc{bounds.state.low}\tc{ = [0;-inf;0;-inf];} sets a bound only for the first and third element of the state vector. All entries relating to time are scalar, entries relating to state $\bm{x}$ are column vectors of length $N_x$, and entries relating to control $\bm{u}$ are column vectors of length $N_u$.
\begin{itemize} \setlength\itemsep{-0.1em}
\item \tc{problem.}\hc{bounds.initialTime.low}\tc{ = }$t_0^-$ 
\item \tc{problem.}\hc{bounds.initialTime.upp}\tc{ = }$t_0^+$ 
\item \tc{problem.}\hc{bounds.finalTime.low}\tc{ = }$t_F^-$ 
\item \tc{problem.}\hc{bounds.finalTime.upp}\tc{ = }$t_F^+$ 
\item \tc{problem.}\hc{bounds.initialState.low}\tc{ = }$\bm{x}_0^-$ 
\item \tc{problem.}\hc{bounds.initialState.upp}\tc{ = }$\bm{x}_0^+$ 
\item \tc{problem.}\hc{bounds.finalState.low}\tc{ = }$\bm{x}_F^-$ 
\item \tc{problem.}\hc{bounds.finalState.upp}\tc{ = }$\bm{x}_F^+$ 
\item \tc{problem.}\hc{bounds.state.low}\tc{ = }$\bm{x}^-$ 
\item \tc{problem.}\hc{bounds.state.upp}\tc{ = }$\bm{x}^+$ 
\item \tc{problem.}\hc{bounds.control.low}\tc{ = }$\bm{u}^-$ 
\item \tc{problem.}\hc{bounds.control.upp}\tc{ = }$\bm{u}^+$ 
\end{itemize}
 

\subsubsection*{\tc{problem}.\hc{guess}}

The guess struct provides the optimization with an initialization. All fields are mandatory. Internally, \texttt{trajOpt} uses the guess struct to determine the dimension of the state and control. The number of grid points in the guess struct does not correspond to the number of grid points in the solution. Instead, \texttt{trajOpt} constructs the solution grid using information from the options struct, and then uses interpolation of the data in guess to evaluate the initial value of the solution grid.

\texttt{size($t$) = [1, $N_t$]}, \\
\texttt{size($\bm{x}$) = [$N_x$, $N_t$]}, \\
\texttt{size($\bm{u}$) = [$N_u$, $N_t$]}, \\

\begin{itemize} \setlength\itemsep{-0.1em}
\item \tc{problem.}\hc{guess.time}\tc{ = }$t_g \qquad$ \texttt{size($t_g$) = [1, $N_g$]}   
\item \tc{problem.}\hc{guess.state}\tc{ = }$\bm{x}_g \qquad$ \texttt{size($\bm{x}_g$) = [$N_x$, $N_g$]}  
\item \tc{problem.}\hc{guess.control}\tc{ = }$\bm{u}_g \qquad$ \texttt{size($\bm{u}_g$) = [$N_u$, $N_g$]}
\end{itemize}


\subsubsection*{\tc{problem}.\hc{options}}

The options struct provides options for both \texttt{trajOpt} and \texttt{fmincon}, which is called by \texttt{trajOpt} to solve the underlying nonlinear program. All fields, including \hc{options} itself may be omitted. If \tc{problem}.\hc{options} is a struct array, then \texttt{trajOpt} will run a sequence of trajectory optimizations, one for each element of the options struct array. The solution to each sub-problem is used to initialize the following. This is used for manual mesh refinement, and is illustrated in several of the example problems in the \texttt{demo/} directory. 

\par The \texttt{trapezoid}, \texttt{hermiteSimpson}, and \texttt{rungeKutta} methods work without any external dependencies. The \texttt{chebyshev} method is written entirely in \texttt{trajOpt}, but relies on ChebFun\cite{Driscoll2014} for computing some of the low-level implementation details for the orthogonal polynomials. ChebFun is easy to download and install from \texttt{https://github.com/chebfun/chebfun}. The final method, \texttt{gpops} is actually just a wrapper for the professional software GPOPS-II\cite{Patterson2013}, for which a license can be obtained from \texttt{http://www.gpops2.com/}.

\par Each method has a single field devoted to it in the options struct. The options in each of these fields are used exclusively by the method for which they are named. This allows for method-specific options, like the number of sub-steps in the multiple shooting method. The field \tc{problem}.\hc{options.gpops} is passed to GPOPS-II as the \tc{setup} struct, allowing the user to specify low-level options in GPOPS-II. 

\begin{itemize} \setlength\itemsep{-0.1em}
\item \tc{problem.}\hc{options.nlpOpt} = a struct of options that are passed directly to fmincon
\item \tc{problem.}\hc{options.verbose} = how much detail to provide? \\
$0\to$ no printing,  $1\to$ default, $2\to$ extra warnings, $3\to$ debug. 
\item \tc{problem.}\hc{options.defaultAccuracy} = used to set the default settings for all methods \\
possible values:  \texttt{ 'low' } $\quad$ \texttt{ 'medium' } $\quad$ \texttt{'high'}
\item \tc{problem.}\hc{options.method} = a string, specifying which method to use. 
	\begin{itemize} \setlength\itemsep{-0.1em}
	\item \hc{'trapezoid'} $\quad \to \quad $ trapezoidal direct collocation
	\item \hc{'hermiteSimpson'} $\quad \to \quad $ Hermite--Simpson direct collocation
	\item \hc{'chebyshev'} $\quad \to \quad $ Chebyshev--Lobatto orthogonal collocation
	\item \hc{'rungeKutta'} $\quad \to \quad $ $4^\text{th}$-order Runge--Kutta Multiple Shooting
	\item \hc{'gpops'} $\quad \to \quad $ wrapper for calling GPOPS-II
	\end{itemize}
\item \tc{problem.}\hc{options.trapezoid.nGrid} = number of grid-points in \texttt{trapezoid}
\item \tc{problem.}\hc{options.hermiteSimpson.nSegment} = number of segments in \texttt{hermiteSimpson} 
\item \tc{problem.}\hc{options.chebyshev.nColPts} = number of collocation points in \texttt{chebyshev} 
\item \tc{problem.}\hc{options.rungeKutta.nSegment} = number of segments in \texttt{rungeKutta}
\item \tc{problem.}\hc{options.rungeKutta.nSubStep} = number of sub-steps per segment in \texttt{rungeKutta}
\item \tc{problem.}\hc{options.gpops} = low-level options for \texttt{gpops}  (passed like \texttt{setup})
\end{itemize}


\section{Technical Details}



\subsection{Problem Statement}

Minimize the objective function:
\begin{equation*}
\underset{t_0, t_F, \bm{x}(t), \bm{u}(t)} \min \;
J_B\big(t_0,t_F,\bm{x}(t_0),\bm{x}(t_F) \big) + 
\int_{t_0}^{t_F} \! J_P \big( \tau, \bm{x}(\tau), \bm{u}(\tau) \big)  \; d\tau
\label{eqn:optimTraj_objectiveFunction}
\end{equation*}

Subject to the constraints:
\begin{align*}
& \quad   \dot{\bm{x}}(t) = \bm{f} \big(t, \bm{x}(t), \bm{u}(t)\big)   & \quad & \eqnTxt{system dynamics}\\

& \quad   \bm{C}_P \big(t, \bm{x}(t), \bm{u}(t)\big) \leq \bm{0}    & \quad &   \eqnTxt{path constraints}\\

& \quad   \bm{C}_B\big(t_0,t_F,\bm{x}(t_0),\bm{x}(t_F) \big) \leq \bm{0}    & \quad & \eqnTxt{boundary constraints} \\

& \quad   \bm{x}^- \leq \bm{x}(t) \leq \bm{x}^+    & \quad & \eqnTxt{constant bounds on state} \\
& \quad   \bm{u}^- \leq \bm{u}(t) \leq \bm{u}^+    & \quad & \eqnTxt{constant bounds on control} \\

& \quad   t^- \leq t_0 < t_F \leq t^+    & \quad & \eqnTxt{bounds on initial and final time} \\
& \quad   \bm{x}_0^- \leq \bm{x}(t_0) \leq  \bm{x}_0^+   & \quad & \eqnTxt{bound on initial state} \\
& \quad   \bm{x}_F^- \leq \bm{x}(t_F) \leq \bm{x}_F^+    & \quad & \eqnTxt{bound on final state}
\end{align*}

Assuming that the user-defined objective, dynamics, and constraint functions ($J_P$, $J_B$, $\bm{f}$, $\bm{C}_P$, $\bm{C}_B$) are smooth. The user must provide an initial guess for the decision variables $t_0$, $t_F$, $\bm{x}(t)$, and $\bm{u}(t)$, and ensure that a feasible solution exists.

\subsection{Trapezoidal Direct Collocation:  \hc{trapezoid}}

Trapezoidal direct collocation works by making the assumption that the optimal trajectory can be approximated using a low-order spline. In this case, the dynamics, objective function, and control trajectories are approximated using a linear spline, and the state trajectory is the quadratic spline, obtained by integration of the (linear) dynamics spline. Integration of a linear spline is computed using the trapezoid rule, hence the name. The implementation of trapezoidal direct collocation in OptimTraj is almost entirely based on the method as it is described in \cite{Betts2010}.

\subsection{Hermite-Simpson Direct Collocation:  \hc{hermiteSimpson}}

Hermite-Simpson direct collocation works by making the assumption that the optimal trajectory can be approximated using a medium-order spline. In this case, the dynamics, objective function, and control trajectories are approximated using a quadratic spline, and the state trajectory is a cubic hermite spline, obtained by integration of the (quadratic) dynamics spline. Integration of a quadratic spline is computed using Simpson's rule. There are two versions of this method: separated and compressed. In the separated form, the state at the mid-point of each trajectory segment is included as a decision variable, and the Hermite-interpolation is enforced with a constraint. In the compressed form, the state at the mid-point is computed from the definition of the Hermite-interpolant during optimization. OptimTraj implements the separated Hermite-Simpson method, almost entirely based on the method as it is described in \cite{Betts2010}.

\subsection{Chebyshev--Lobatto Orthogonal Collocation:  \hc{chebyshev}}

Chebyshev--Lobatto orthogonal collocation works by representing the entire trajectory using a high-order Chebyshev orthogonal polynomial. The implementation here might also be called pseudospectral or global collocation, since the entire trajectory is represented using a single segment, rather than several segments. Orthogonal collocation methods can be divided into three categories: Gauss, Radau, and Lobatto. In a Gauss method, neither end-point of a segment is a collocation point, in a Radau method, a single end-point of the segment is a collocation point, and in a Lobatto method, both end-points are collocation points \cite{Garg2010}.

\par The implementation here uses the ChebFun toolbox \cite{Driscoll2014} for computing the Chebyshev--Lobatto collocation points, and also for interpolation of the solution. More details regarding orthogonal polynomials and calculations with them can be found in \cite{Berrut2004a} and \cite{Trefethen2012}. The details of the collocation method itself are largely drawn from \cite{Vlassenbroeck1988}.

\subsection{Runge--Kutta $4^\text{th}$-order Multiple Shooting:  \hc{rungeKutta}}

A multiple shooting method works by breaking the trajectory into segments, and approximating each segment using an explicit simulation. In this case, we use a $4^\text{th}$-order Runge--Kutta method for the simulation. A defect constraint is used to ensure that the end of each trajectory segment correctly lines up with the next. The interpolation of the solution trajectory, for both control and state, is approximated using a cubic Hermite spline. The implementation here, except for interpolation, is as described in \cite{Betts2010}.

\subsection{GPOPS-II:    \hc{gpops}}

OptimTraj includes a wrapper to the software GPOPS-II \cite{Patterson2013}, a professionally developed trajectory optimization library for Matlab. It implements a nicely optimized version of Radau orthogonal collocation with adaptive meshing \cite{Darby2011a}. There are a few choice in GPOPS-II for both collocation metho and the adaptive meshing. It also supports analytic gradients using automatic differentiation. It is included in OptimTraj for two reasons: 1) GPOPS-II can be used to benchmark and verify the methods in OptimTraj, and 2) GPOPS-II provides a collection of methods that are not otherwise available in OptimTraj.

\subsection{Resources for Learning Trajectory Optimization}

The single best resource for learning about trajectory optimization is the textbook by John T. Betts: "Practical Methods for Optimal Control and Estimation Using Nonlinear Programming" \cite{Betts2010}. The textbook by Bryson and Ho: "Applied Optimal Control" \cite{Bryson1975}. Both of these books are also excellent for learning about nonlinear programming and optimization.\\

\par There are two good review papers about trajectory optimization. The paper by Betts \cite{Betts1998} is more focused on direct collocation and shooting methods, while the paper by Rao \cite{Rao2009} is more focused on orthogonal collocation methods.\\

\par Understanding orthogonal collocation is rather challenging, and I found that it was first necessary to get a solid understanding of orthogonal polynomials and function approximation. I started by reading "Approximation Theory and Approximation Practice" \cite{Trefethen2012} by Trefethen, and then his paper on barycentric interpolation \cite{Berrut2004a}. Some intuition can also be obtained by reading the source code of ChebFun \cite{Driscoll2014}: \\
\url{http://www.chebfun.org/}\\

\par Russ Tedrake at MIT has also provided some excellent resources for learning about trajectory optimization, and robotics in general. The first is his online course on underactuated robotics. You can download the pdf of the course notes \cite{Tedrake2009}, or access the full course and video lectures at: \\
\href{http://ocw.mit.edu/courses/electrical-engineering-and-computer-science/6-832-underactuated-robotics-spring-2009/index.htm}{http://ocw.mit.edu/courses/} \\

Second, his group at MIT are producing a planning, control, and analysis toolbox called Drake, which includes trajectory optimization. It is open source, so it is another good place to read the source code and figure out how it works: \\
\url{https://github.com/RobotLocomotion/drake}\\

\par There is also a tutorial page on my website. It contains a high-level summary, useful links, visualization of single shooting, multiple shooting, and collocation, and a link to a simple multiple shooting tutorial in Matlab. \\
\url{http://www.matthewpeterkelly.com/tutorials/trajectoryOptimization/index.html}

%=========================================================================
\pagebreak
\bibliographystyle{abbrv} \bibliography{TrajectoryOptimization}

%=================================================
\end{document}