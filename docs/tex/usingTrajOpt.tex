\section{Using TrajOpt}

There is a single calling sequence for using TrajOpt:  \tc{soln = }\hc{trajOpt}\tc{(problem)}. The input \tc{problem} is a struct that describes a trajectory optimization problem, and the output \tc{soln} is a struct that gives details regarding the solution to the trajectory optimization problem. This section is similar to the help file for \tc{trajOpt()}.

\subsection{Notation}

Throughout this section we will use $t$ for time, $\bm{x}$ for state, and $\bm{u}$ for control. We will use $N$ for an integer, where $N_t$ is the number of grid-points along the trajectory and $N_g$ is the number of grid-points in the initial guess. The dimension of the state is given by $N_x$ and the dimension of the control is given by $N_u$. We will use this notation: \tc{problem.}\hc{func} to indicate that the field \hc{func} must be typed exactly as shown, while the struct \tc{problem} can be named anything.

\par We will use $t_0$ and $t_F$ to indicate the initial and final times. Similarly, $\bm{x}_0 = \bm{x}(t_0)$ and $\bm{x}_F = \bm{x}(t_F)$ are the initial and final state. Throughout the software, we use the convention that time is a row vector, while the dimensions of state and control are a column vector. Thus, \\
\texttt{size($t$) = [1, $N_t$]}, \\
\texttt{size($\bm{x}$) = [$N_x$, $N_t$]}, \\
\texttt{size($\bm{u}$) = [$N_u$, $N_t$]}, \\
\texttt{size($\bm{x}_0$) = size($\bm{x}_F$)  = [$N_x$, 1]}. 


\subsection{Constructing the Input to TrajOpt}

The input to \hc{trajOpt()} is a single struct, which we will call the \tc{problem} struct. The \tc{problem} struct has four fields. The \tc{problem}.\hc{func} struct contains a set of user-defined function handles to the dynamics, objective, and constraint functions. The \tc{problem}.\hc{bounds} struct contains all constant bounds on trajectory (time, state, and control). The \tc{problem}.\hc{guess} struct contains an initial guess for the trajectory. Finally, the \tc{problem}.\hc{options} struct contains options for both \hc{trajOpt()} and \hc{fmincon()}, which it calls to solve the underlying non-linear program.

\subsubsection*{\tc{problem}.\hc{func}}

There are five fields in the \tc{problem}.\hc{func} struct, each of which is a user-defined function handle. The only mandatory field is \texttt{func.dynamics}, and at least one of either \texttt{func.pathObj} or \texttt{func.bndObj}. All other fields can simply be omitted or left empty \texttt{[]}. Here we will list the prototype for each function handle: \texttt{outArgs = funName(inArgs)}. The user can pass additional parameters using the Matlab anonymous function syntax: \texttt{funHandle = @( funName(inArgs, extraParams) );}.

\par Both of the direct collocation methods (trapezoidal and Hermite-Simpson) support analytic gradients. The details regarding the construction of these function handles is provided in the help files for each of these methods: \texttt{>> help trapezoid} and \texttt{>> help hermiteSimpson}. If you're using analytic gradients, then you should look at the \texttt{demo/gradientsTutorial} for an simple tutorial example and \texttt{demo/fiveLinkBiped} for a complicated example.

\begin{itemize} \setlength\itemsep{-0.1em}
\item \tc{problem.}\hc{func.dynamics}
	${\bm: \quad \to \quad }$
	\tc{$\bm{\dot{x}}$ = }\hc{dynamics}\tc{($t$,$\bm{x}$,$\bm{u}$)}
	\\ where $\bm{\dot{x}}$ is the time derivative of $\bm{x}$ and \texttt{size($\bm{\dot{x}}$)==size($\bm{x}$)}
\item \tc{problem.}\hc{func.pathObj}
	${\bm: \quad \to \quad }$
	\tc{$\bm{J_P}$ = }\hc{pathObj}\tc{($t$,$\bm{x}$,$\bm{u}$)}
	\\ where $\bm{J_P}$ is the integrand of the objective function and \texttt{size($\bm{J_P}$)==size($t$)}
\item \tc{problem.}\hc{func.pathCst}
	${\bm: \quad \to \quad }$
	\tc{$\bm{[C_P^a, C_P^b ]}$ = }\hc{pathCst}\tc{($t$,$\bm{x}$,$\bm{u}$)}
	\\ where $\bm{C_P^a}=\bm{0}$ is path equality constraint, and $\bm{C_P^b} \leq \bm{0}$ is the path inequality constraint. Either may be left empty \texttt{[ ]}. 	 	The 	number of columns in each must equal that of time, but the number of rows in each is arbitrary (it just must be consistent between function calls).
\item \tc{problem.}\hc{func.bndObj}
	${\bm: \quad \to \quad }$
	\tc{$J_B$ = }\hc{bndObj}\tc{($t_0$,$\bm{x}_0$,$t_F$,$\bm{x}_F$)}
	\\ where $\bm{J_B}$ is a scalar cost associated with the boundary points of the trajectory. 
\item \tc{problem.}\hc{func.bndCst}
	${\bm: \quad \to \quad }$
	\tc{$\bm{[C_B^a, C_B^b] }$ = }\hc{bndCst}\tc{($t_0$,$\bm{x}_0$,$t_F$,$\bm{x}_F$)}
	\\ where $\bm{C_B^a}=\bm{0}$ is boundary equality constraint, and $\bm{C_B^b} \leq \bm{0}$ is the boundary inequality constraint. Either may be left empty \texttt{[ ]}. Each is a column vector of arbitrary length, provided that it is consistent between function calls.
\end{itemize}


\subsubsection*{\tc{problem}.\hc{bounds}}

The bounds struct provides constant bounds on the state and control along the trajectory, as well as the time and state on the boundaries. All fields are either scalar or a column vector, and can be omitted (or left empty) if not needed. If you need to include a bound on only part of the state or control, then set the remaining entries to $\pm\infty$. For example: \hc{bounds.state.low}\tc{ = [0;-inf;0;-inf];} sets a bound only for the first and third element of the state vector. All entries relating to time are scalar, entries relating to state $\bm{x}$ are column vectors of length $N_x$, and entries relating to control $\bm{u}$ are column vectors of length $N_u$.
\begin{itemize} \setlength\itemsep{-0.1em}
\item \tc{problem.}\hc{bounds.initialTime.low}\tc{ = }$t_0^-$ 
\item \tc{problem.}\hc{bounds.initialTime.upp}\tc{ = }$t_0^+$ 
\item \tc{problem.}\hc{bounds.finalTime.low}\tc{ = }$t_F^-$ 
\item \tc{problem.}\hc{bounds.finalTime.upp}\tc{ = }$t_F^+$ 
\item \tc{problem.}\hc{bounds.initialState.low}\tc{ = }$\bm{x}_0^-$ 
\item \tc{problem.}\hc{bounds.initialState.upp}\tc{ = }$\bm{x}_0^+$ 
\item \tc{problem.}\hc{bounds.finalState.low}\tc{ = }$\bm{x}_F^-$ 
\item \tc{problem.}\hc{bounds.finalState.upp}\tc{ = }$\bm{x}_F^+$ 
\item \tc{problem.}\hc{bounds.state.low}\tc{ = }$\bm{x}^-$ 
\item \tc{problem.}\hc{bounds.state.upp}\tc{ = }$\bm{x}^+$ 
\item \tc{problem.}\hc{bounds.control.low}\tc{ = }$\bm{u}^-$ 
\item \tc{problem.}\hc{bounds.control.upp}\tc{ = }$\bm{u}^+$ 
\end{itemize}
 

\subsubsection*{\tc{problem}.\hc{guess}}

The guess struct provides the optimization with an initialization. All fields are mandatory. Internally, \texttt{trajOpt} uses the guess struct to determine the dimension of the state and control. The number of grid points in the guess struct does not correspond to the number of grid points in the solution. Instead, \texttt{trajOpt} constructs the solution grid using information from the options struct, and then uses interpolation of the data in guess to evaluate the initial value of the solution grid.

\begin{itemize} \setlength\itemsep{-0.1em}
\item \tc{problem.}\hc{guess.time}\tc{ = }$t_g \qquad$ \texttt{size($t_g$) = [1, $N_g$]}   
\item \tc{problem.}\hc{guess.state}\tc{ = }$\bm{x}_g \qquad$ \texttt{size($\bm{x}_g$) = [$N_x$, $N_g$]}  
\item \tc{problem.}\hc{guess.control}\tc{ = }$\bm{u}_g \qquad$ \texttt{size($\bm{u}_g$) = [$N_u$, $N_g$]}
\end{itemize}



\subsubsection*{\tc{problem}.\hc{options}}

The options struct provides options for both \texttt{trajOpt} and \texttt{fmincon}, which is called by \texttt{trajOpt} to solve the underlying nonlinear program. All fields, including \hc{options} itself may be omitted. If \tc{problem}.\hc{options} is a struct array, then \texttt{trajOpt} will run a sequence of trajectory optimizations, one for each element of the options struct array. The solution to each sub-problem is used to initialize the following. This is used for manual mesh refinement, and is illustrated in several of the example problems in the \texttt{demo/} directory. 

\par The \texttt{trapezoid}, \texttt{hermiteSimpson}, and \texttt{rungeKutta} methods work without any external dependencies. The \texttt{chebyshev} method is written entirely in \texttt{trajOpt}, but relies on ChebFun\cite{Driscoll2014} for computing some of the low-level implementation details for the orthogonal polynomials. ChebFun is easy to download and install from \texttt{https://github.com/chebfun/chebfun}. The final method, \texttt{gpops} is actually just a wrapper for the professional software GPOPS-II\cite{Patterson2013}, for which a license can be obtained from \texttt{http://www.gpops2.com/}.

\par Each method has a single field devoted to it in the options struct. The options in each of these fields are used exclusively by the method for which they are named. This allows for method-specific options, like the number of sub-steps in the multiple shooting method. The field \tc{problem}.\hc{options.gpops} is passed to GPOPS-II as the \tc{setup} struct, allowing the user to specify low-level options in GPOPS-II. 

\begin{itemize} \setlength\itemsep{-0.1em}
\item \tc{problem.}\hc{options.nlpOpt} = a struct of options that are passed directly to fmincon
\item \tc{problem.}\hc{options.verbose} = how much detail to provide? \\
$0\to$ no printing,  $1\to$ default, $2\to$ extra warnings, $3\to$ debug. 
\item \tc{problem.}\hc{options.defaultAccuracy} = used to set the default settings for all methods \\
possible values:  \texttt{ 'low' } $\quad$ \texttt{ 'medium' } $\quad$ \texttt{'high'}
\item \tc{problem.}\hc{options.method} = a string, specifying which method to use. 
	\begin{itemize} \setlength\itemsep{-0.1em}
	\item \hc{'trapezoid'} $\quad \to \quad $ trapezoidal direct collocation
	\item \hc{'hermiteSimpson'} $\quad \to \quad $ Hermite--Simpson direct collocation
	\item \hc{'chebyshev'} $\quad \to \quad $ Chebyshev--Lobatto orthogonal collocation
	\item \hc{'rungeKutta'} $\quad \to \quad $ $4^\text{th}$-order Runge--Kutta Multiple Shooting
	\item \hc{'gpops'} $\quad \to \quad $ wrapper for calling GPOPS-II
	\end{itemize}
\item \tc{problem.}\hc{options.trapezoid.nGrid} = number of grid-points in \texttt{trapezoid}
\item \tc{problem.}\hc{options.hermiteSimpson.nSegment} = number of segments in \texttt{hermiteSimpson} 
\item \tc{problem.}\hc{options.chebyshev.nColPts} = number of collocation points in \texttt{chebyshev} 
\item \tc{problem.}\hc{options.rungeKutta.nSegment} = number of segments in \texttt{rungeKutta}
\item \tc{problem.}\hc{options.rungeKutta.nSubStep} = number of sub-steps per segment in \texttt{rungeKutta}
\item \tc{problem.}\hc{options.gpops} = low-level options for \texttt{gpops}  (passed like \texttt{setup})
\end{itemize}



\section{Details for the Output of TrajOpt}
The output of \hc{trajOpt()} is a single struct, which we will call the \tc{soln} struct. The \tc{soln} struct has four fields. The \tc{soln}.\hc{grid} struct contains value of the trajectory along the exact grid-points that were used by the non-linear program.  The \tc{soln}.\hc{interp} struct contains a set of function handles that are used for method-consistent interpolation the solution trajectory. The \tc{soln}.\hc{info} struct contains a variety of details about the solution to the trajectory optimizaiton problem. Finally, the \tc{soln}.\hc{problem} struct is a copy of the input to \texttt{trajOpt}, including all default options that were used.


\subsubsection*{\tc{soln}.\hc{grid}}
The grid struct contains trajectory evaluated at the grid-points used by the transcriptions algorithm. The spacing and number of the grid-points is dependent on the transcription method and options that were passed to it.
\begin{itemize} \setlength\itemsep{-0.1em}
\item \tc{soln.}\hc{grid.time}\tc{ = }$t \qquad$ \texttt{size($t$) = [1, $N_t$]}   
\item \tc{soln.}\hc{grid.state}\tc{ = }$\bm{x} \qquad$ \texttt{size($\bm{x}$) = [$N_x$, $N_t$]}  
\item \tc{soln.}\hc{grid.control}\tc{ = }$\bm{u} \qquad$ \texttt{size($\bm{u}$) = [$N_u$, $N_t$]}
\end{itemize}


\subsubsection*{\tc{soln}.\hc{interp}}
The interp struct contains two function handles, one for the state and another for control. Each function handle takes a single argument (time), and returns either the state or control vector at each point in the time vector. The interpolation method is consistent with whatever transcription method was used. For example, if you used the \texttt{trapezoid} method, then \tc{soln}.\hc{interp.state} will use linear interpolation between the control grid points. This method-consistent interpolation is particularly important for high-order methods, such as \texttt{chebyshev}. Note: in both cases below, the function name \texttt{'specialInterpolation'} is just a placeholder. It will be different for each method.
\begin{itemize}
\item \tc{soln.}\hc{interp.state}
	${\bm: \quad \to \quad }$
	\tc{$\bm{x}$ = }\tc{specialInterpolation($t$)}
	\\ where $\bm{x} = \bm{x}(t)$ is the value of the state at each point in $t$ and \texttt{size($\bm{x}$)==[$N_x$, length($t$)]}
\item \tc{soln.}\hc{interp.control}
	${\bm: \quad \to \quad }$
	\tc{$\bm{u}$ = }\tc{specialInterpolation($t$)}
	\\ where $\bm{u} = \bm{u}(t)$ is the value of the state at each point in $t$ and \texttt{size($\bm{u}$)==[$N_u$, length($t$)]}
\item \tc{soln.}\hc{interp.collCst}
	${\bm: \quad \to \quad }$
	\tc{$\bm{\varepsilon}$ = }\tc{specialInterpolation($t$)}
	\\ where $\bm{\varepsilon} = \bm{\varepsilon}(t)$ is the value of the collocation constraint. This will be zero at each collocation point and non-zero elsewhere. It is used as an error metric. Currently only available for the \texttt{trapezoid} and \texttt{hermiteSimpson} methods.  \texttt{size($\bm{\varepsilon}$)==size($\bm{x}$)}
\end{itemize}



\subsubsection*{\tc{soln}.\hc{info}}

The info struct provides details regarding the performance of both \texttt{trajOpt} and \texttt{fmincon}, as well as general information about the solution. All solvers return the fields described here, with two exceptions. First, many of the methods output additional information, such as error metrics for the direct collocation methods. Type \texttt{>>help [method name]} for details. Second, the \texttt{gpops} method does not use \texttt{fmincon}. Instead of returning the information from \texttt{fmincon}, the entire output of GPOPS-II is placed in the field \tc{soln}.\hc{info.gpops}.

\par If the \tc{problem}.\hc{options} is a struct array, then \tc{soln} will also be a struct array, with one solution for each of the sub-problems specified by the options struct.

\begin{itemize} \setlength\itemsep{-0.1em}
\item \tc{soln.}\hc{info.nlpTime} = total time spent in the nonlinear programming solver.
\item \tc{soln.}\hc{info.exitFlag} = exit flag returned by the nonlinear programmign solver.
\item \tc{soln.}\hc{info.objVal} = value of the objective function
\item \tc{soln.}\hc{info.*} = all fields from the \texttt{fmincon problem struct} 
\item \tc{soln.}\hc{info.error} = integral of absolute error in collocation constraint along each segment of the trajectory. Currently only available for the \texttt{trapezoid} and \texttt{hermiteSimpson} methods. 
\item \tc{soln.}\hc{info.maxError} = \texttt{max(max(\tc{soln.}\hc{info.error}))}
\end{itemize}

\subsubsection*{\tc{soln}.\hc{problem}}

The problem struct is similar to the problem struct that was passed in, but \texttt{trajOpt} fills in all omitted fields with the defaults that were actually used by the program.